\documentclass[aspectratio=169]{beamer}

% Tema base
\usetheme{metropolis}
\usepackage{insper-mtheme}

\usepackage{appendixnumberbeamer}

\usepackage{booktabs}
\usepackage[scale=2]{ccicons}

\usepackage{pgfplots}
\usepgfplotslibrary{dateplot}

\usepackage{xspace}
\newcommand{\themename}{\textbf{\textsc{metropolis}}\xspace}

% --- CUSTOMIZAÇÃO DA CAPA ---
\setbeamertemplate{title page}
{%
  \begin{tikzpicture}[remember picture,overlay]
    % PDF de fundo ocupa toda a página
    \node[at=(current page.center)] {
      \includegraphics[width=\paperwidth,height=\paperheight]{fundo_capa.pdf}
    };

    % --- Texto em posições absolutas ---

    % Título
    \node[anchor=west,text width=0.6\paperwidth] 
      at ([xshift=1.3cm,yshift=5.5cm]current page.south west)
      {\usebeamerfont{title}\inserttitle};

    % Autores
    \node[anchor=west,text width=0.6\paperwidth] 
      at ([xshift=1.3cm,yshift=3.5cm]current page.south west)
      {\usebeamerfont{author}\insertauthor};

    % Data
    \node[anchor=east] 
      at ([xshift=-0.8cm,yshift=2.0cm]current page.south east)
      {\usebeamerfont{date}\insertdate};

    % Centro/Núcleo
    \node[anchor=east,font=\bfseries] 
      at ([xshift=-0.8cm,yshift=1.2cm]current page.south east)
      {\insertinstitute};
  \end{tikzpicture}
}

% --- DADOS DO USUÁRIO ---
\title{Título da Pesquisa e/ou relatório elaborado}
\author{Autor 1\\
    Autor 2\\
    Autor 3\\
    Autor 4}
\date{24 de março, 2025}
\institute{Nome do centro/núcleo/iniciativa}

\begin{document}

\begin{frame}
  \titlepage
\end{frame}

\begin{frame}{Sumário}
  \setbeamertemplate{section in toc}[sections numbered]
  \tableofcontents[hideallsubsections]
\end{frame}

\section{Introdução}

\begin{frame}[fragile]{Metropolis}

  O tema \themename é um tema do Beamer com ruído visual mínimo,
  inspirado no \href{https://github.com/hsrmbeamertheme/hsrmbeamertheme}{Tema Beamer \textsc{hsrm}} de Benjamin Weiss.

  Habilite o tema carregando:

  \begin{verbatim}    \documentclass{beamer}
    \usetheme{metropolis}\end{verbatim}

  Note que você precisa ter a fonte \emph{Fira Sans} da Mozilla e o XeTeX
  instalados para aproveitar essa tipografia maravilhosa.
\end{frame}

\begin{frame}[fragile]{Seções}
  Seções agrupam slides do mesmo tópico

  \begin{verbatim}    \section{Elementos}\end{verbatim}

  para os quais o \themename fornece um bom indicador de progresso \ldots
\end{frame}

\section{Formatos de Título}

\begin{frame}{Formatos de título do Metropolis}
	\themename suporta 4 formatos de título diferentes:
	\begin{itemize}
		\item Regular
		\item \textsc{Versalete}
		\item \textsc{versalete completo}
		\item MAIÚSCULAS
	\end{itemize}
	Eles podem ser definidos de forma global para todos os tipos de título ou individualmente.
\end{frame}

{
    \metroset{titleformat frame=smallcaps}
\begin{frame}{Versalete}
	Este slide usa o formato de título \texttt{smallcaps}.

	\begin{alertblock}{Possíveis Problemas}
		Atenção: nem todas as fontes suportam versaletes. Se, por exemplo, você compilar sua apresentação com pdfTeX e a fonte Computer Modern Sans Serif, todo texto em versalete será renderizado com a fonte Computer Modern Serif.
	\end{alertblock}
\end{frame}
}

{
\metroset{titleformat frame=allsmallcaps}
\begin{frame}{Versalete Completo}
	Este slide usa o formato de título \texttt{allsmallcaps}.

	\begin{alertblock}{Possíveis Problemas}
		Como esse formato também usa versaletes, você enfrentará os mesmos problemas que com o formato \texttt{smallcaps}. Além disso, esse formato pode causar outros problemas. Consulte a documentação se decidir usá-lo.

		Regra geral: use-o apenas para títulos com texto puro.
	\end{alertblock}
\end{frame}
}

{
\metroset{titleformat frame=allcaps}
\begin{frame}{MAIÚSCULAS}
	Este slide usa o formato de título \texttt{allcaps}.

	\begin{alertblock}{Possíveis Problemas}
		Este formato de título não é tão problemático quanto o formato \texttt{allsmallcaps}, mas basicamente sofre das mesmas limitações. Consulte a documentação se quiser utilizá-lo.
	\end{alertblock}
\end{frame}
}

\section{Elementos}

\begin{frame}[fragile]{Tipografia}
      \begin{verbatim}O tema fornece padrões razoáveis para
\emph{enfatizar} texto, \alert{destacar} partes
ou mostrar \textbf{resultados em negrito}.\end{verbatim}

  \begin{center}torna-se\end{center}

  O tema fornece padrões razoáveis para \emph{enfatizar} texto,
  \alert{destacar} partes ou mostrar \textbf{resultados em negrito}.
\end{frame}

\begin{frame}{Teste de fontes}
  \begin{itemize}
    \item Regular
    \item \textit{Itálico}
    \item \textsc{Versalete}
    \item \textbf{Negrito}
    \item \textbf{\textit{Negrito Itálico}}
    \item \textbf{\textsc{Negrito Versalete}}
    \item \texttt{Monoespaçado}
    \item \texttt{\textit{Monoespaçado Itálico}}
    \item \texttt{\textbf{Monoespaçado Negrito}}
    \item \texttt{\textbf{\textit{Monoespaçado Negrito Itálico}}}
  \end{itemize}
\end{frame}

\begin{frame}{Listas}
  \begin{columns}[T,onlytextwidth]
    \column{0.33\textwidth}
      Itens
      \begin{itemize}
        \item Leite \item Ovos \item Batatas
      \end{itemize}

    \column{0.33\textwidth}
      Enumerações
      \begin{enumerate}
        \item Primeiro, \item Segundo e \item Último.
      \end{enumerate}

    \column{0.33\textwidth}
      Descrições
      \begin{description}
        \item[PowerPoint] Meeh. \item[Beamer] Yeeeha.
      \end{description}
  \end{columns}
\end{frame}

\begin{frame}{Animação}
  \begin{itemize}[<+- | alert@+>]
    \item \alert<4>{Isto é\only<4>{ realmente} importante}
    \item Agora isto
    \item E agora isto
  \end{itemize}
\end{frame}

\begin{frame}{Figuras}
  \begin{figure}
    \newcounter{density}
    \setcounter{density}{20}
    \begin{tikzpicture}
      \def\couleur{alerted text.fg}
      \path[coordinate] (0,0)  coordinate(A)
                  ++( 90:5cm) coordinate(B)
                  ++(0:5cm) coordinate(C)
                  ++(-90:5cm) coordinate(D);
      \draw[fill=\couleur!\thedensity] (A) -- (B) -- (C) --(D) -- cycle;
      \foreach \x in {1,...,40}{%
          \pgfmathsetcounter{density}{\thedensity+20}
          \setcounter{density}{\thedensity}
          \path[coordinate] coordinate(X) at (A){};
          \path[coordinate] (A) -- (B) coordinate[pos=.10](A)
                              -- (C) coordinate[pos=.10](B)
                              -- (D) coordinate[pos=.10](C)
                              -- (X) coordinate[pos=.10](D);
          \draw[fill=\couleur!\thedensity] (A)--(B)--(C)-- (D) -- cycle;
      }
    \end{tikzpicture}
    \caption{Quadrado rotacionado de
    \href{http://www.texample.net/tikz/examples/rotated-polygons/}{texample.net}.}
  \end{figure}
\end{frame}

\begin{frame}{Tabelas}
  \begin{table}
    \caption{Maiores cidades do mundo (fonte: Wikipedia)}
    \begin{tabular}{@{} lr @{}}
      \toprule
      Cidade & População\\
      \midrule
      Cidade do México & 20.116.842\\
      Xangai & 19.210.000\\
      Pequim & 15.796.450\\
      Istambul & 14.160.467\\
      \bottomrule
    \end{tabular}
  \end{table}
\end{frame}

\begin{frame}{Blocos}
  Três tipos de blocos diferentes são pré-definidos e podem ter uma
  cor de fundo opcional.

  \begin{columns}[T,onlytextwidth]
    \column{0.5\textwidth}
      \begin{block}{Padrão}
        Conteúdo do bloco.
      \end{block}

      \begin{alertblock}{Alerta}
        Conteúdo do bloco.
      \end{alertblock}

      \begin{exampleblock}{Exemplo}
        Conteúdo do bloco.
      \end{exampleblock}

    \column{0.5\textwidth}

      \metroset{block=fill}

      \begin{block}{Padrão}
        Conteúdo do bloco.
      \end{block}

      \begin{alertblock}{Alerta}
        Conteúdo do bloco.
      \end{alertblock}

      \begin{exampleblock}{Exemplo}
        Conteúdo do bloco.
      \end{exampleblock}

  \end{columns}
\end{frame}

\begin{frame}{Matemática}
  \begin{equation*}
    e = \lim_{n\to \infty} \left(1 + \frac{1}{n}\right)^n
  \end{equation*}
\end{frame}

\begin{frame}{Gráficos de linha}
  \begin{figure}
    \begin{tikzpicture}
      \begin{axis}[
        mlineplot,
        width=0.9\textwidth,
        height=6cm,
      ]

        \addplot {sin(deg(x))};
        \addplot+[samples=100] {sin(deg(2*x))};

      \end{axis}
    \end{tikzpicture}
  \end{figure}
\end{frame}

\begin{frame}{Gráficos de barras}
  \begin{figure}
    \begin{tikzpicture}
      \begin{axis}[
        mbarplot,
        xlabel={Foo},
        ylabel={Bar},
        width=0.9\textwidth,
        height=6cm,
      ]

      \addplot plot coordinates {(1, 20) (2, 25) (3, 22.4) (4, 12.4)};
      \addplot plot coordinates {(1, 18) (2, 24) (3, 23.5) (4, 13.2)};
      \addplot plot coordinates {(1, 10) (2, 19) (3, 25) (4, 15.2)};

      \legend{lorem, ipsum, dolor}

      \end{axis}
    \end{tikzpicture}
  \end{figure}
\end{frame}

\begin{frame}{Citações}
  \begin{quote}
    Veni, Vidi, Vici
  \end{quote}
\end{frame}

{%
\setbeamertemplate{frame footer}{Meu rodapé personalizado}
\begin{frame}[fragile]{Rodapé do slide}
    \themename define um modelo Beamer personalizado para adicionar texto ao rodapé. Pode ser definido com:
    \begin{verbatim}\setbeamertemplate{frame footer}{Meu rodapé personalizado}\end{verbatim}
\end{frame}
}

\begin{frame}{Referências}
  Algumas referências para demonstração [allowframebreaks] \cite{knuth92,ConcreteMath,Simpson,Er01,greenwade93}
\end{frame}

\section{Conclusão}

\begin{frame}{Resumo}

  Obtenha o código-fonte deste tema e da apresentação de demonstração em:

  \begin{center}\url{github.com/gabs-acc/beamertheme-insper}\end{center}

  O tema \emph{em si} é licenciado sob a
  \href{http://creativecommons.org/licenses/by-sa/4.0/}{Licença Creative Commons
  Atribuição-CompartilhaIgual 4.0 Internacional}.

  \begin{center}\ccbysa\end{center}

\end{frame}

\begin{frame}[standout]
  Perguntas?
\end{frame}

\appendix

\begin{frame}[fragile]{Slides de backup}
  Às vezes é útil adicionar slides ao final da sua apresentação para
  consulta durante perguntas da audiência.

  A melhor forma de fazer isso é incluir o pacote \verb|appendixnumberbeamer|
  no preâmbulo e usar \verb|\appendix| antes dos slides de backup.

  O \themename desativa automaticamente a numeração de slides e a barra de progresso
  para slides no apêndice.
\end{frame}

\begin{frame}[allowframebreaks]{Referências}

  \bibliography{referencias}
  \bibliographystyle{abbrv}

\end{frame}

\end{document}